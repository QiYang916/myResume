\documentclass[11pt,a4paper]{moderncv}

% moderncv themes
%\moderncvtheme[blue]{casual}                 % optional argument are 'blue' (default), 'orange', 'red', 'green', 'grey' and 'roman' (for roman fonts, instead of sans serif fonts)
\moderncvtheme[blue]{classic}                % idem
\usepackage{xunicode, xltxtra}
\XeTeXlinebreaklocale "zh"
\widowpenalty=10000

%\setmainfont[Mapping=tex-text]{文泉驿正黑}

% character encoding
%\usepackage[utf8]{inputenc}                   % replace by the encoding you are using
\usepackage{CJKutf8}
  
% adjust the page margins
\usepackage[scale=0.8]{geometry}
\recomputelengths                             % required when changes are made to page layout lengths
\setmainfont[Mapping=tex-text]{Hiragino Sans GB}
\setsansfont[Mapping=tex-text]{Hiragino Sans GB}
\CJKtilde

% personal data
\firstname{杨奇}
\familyname{}
\title{}               % optional, remove the line if not wanted
\mobile{13738076490}                    % optional, remove the line if not wanted
\email{yangqi916@gmail.com}                      % optional, remove the line if not wanted
%% \quote{\small{``Do what you fear, and the death of fear is certain.''\\-- Anthony Robbins}}

\nopagenumbers{}

\begin{document}

\maketitle

\section{教育}
\cventry{2015 - 至今}{硕士}{浙江大学计算机学院}{将于2018年毕业}{}{}
\cventry{2011 - 2015}{本科}{武汉理工大学计算机学院}{}{}{}  

\section{基本信息}
\cventry{性别}{男}{1992}{}{}{}

\section{项目经历}
\renewcommand{\baselinestretch}{1.2}

\vspace*{0.2\baselineskip}
\cventry{2016}
{知乎日报 仿写 iOS 应用}
{}
{}{}
{知乎日报的第三方iOS客户端,没有第三方库依赖,使用 Objective-C 语言实现。图片轮播, 沉浸式状态栏等基本上都完成了,本个小项目的主要难点是在图片轮播的具体实现。所用技术,MVC,GCD,自定义 tableview cell,图片轮播插件,缓存,UI Animation,Objective-C Runtime 特性}

\vspace*{0.2\baselineskip} 
\cventry{2016}
{storeSearch iOS应用}
{}
{}{}
{根据The iOS Apprentice做的,storeSearch是itunes store 搜索应用,提供了对itunes store的搜索。使用AFNetworking实现网络部分功能。}

\vspace*{0.2\baselineskip}
\cventry{2015}
{网络地址跳变系统}
{SDN, opebdaylight,mininet}
{}{}
{实验室与中兴合作项目,在软件定义网络的环境下,控制器使用opendaylight,网络拓扑使用mininet搭建,opendaylight开发了一个进行地址跳变的模块。本人负责具体实现。在网络拓扑端,研究了mininet源码,使用python编写了网络拓扑搭建脚本等一系列脚本。在控制器端,研究了opendaylight内部各模块的通信机制,根据设计要求使用java编写了相应的地址跳变模块。为opendaylight增加了跳变功能。}

\vspace*{0.2\baselineskip}
\cventry{2016}
{ONOS的相关测试}
{SDN,ONOS,Python}
{研究项目}{}
{设计完成ONOS分布式集群控制器测试方案,测试其的流表安装时延,流表最大安装数量等。本项目至今仍在进行之中。本人负责ONOS的使用与研究,Teston测试框架的使用与研究。}


\renewcommand{\baselinestretch}{1.0}


\section{技能}
\cventry{语言}{C熟练,熟悉OC,c++主要用来刷题,python能读}{}{}{}{}
\cventry{计算机基础}{熟练掌握数据结构和算法,计算机组成原理,操作系统,计算机网络}{}{}{}{}
\cventry{业务能力}{熟悉网络编程,多线程编程。熟练界面搭建,接口调试。熟练debug和重构,致力于生产优雅代码}{}{}{}{}
\cventry{英语}{CET-6 525分,能够熟练阅读各种英文技术文档}{}{}{}{}
\cventry{其它}{有git及Linux使用基础}{}{}{}{}

\section{iOS学习经历}
\cventry{}{iOS的开发(自学)学习了4个月左右,学习过程主要先是看iOS,OC相关的书和资料,接着按 Raywendelich教程边做DEMO边翻官方文档。目前对iOS的各个方面都有基本的认识。知道自己的iOS相关开发经验还不是很多,但迫切希望能够得到一个锻炼学习的机会}{}{}{}{}





\closesection{}                   % needed to renewcommands
\renewcommand{\listitemsymbol}{-} % change the symbol for lists

\end{document}
